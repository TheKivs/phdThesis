\chapter{Visual Scene Analysis}
\minitoc

\begin{abstract}
Building convincing augmented realities requires creating perceptual mappings between an agent and the augmented content of the environment they perceive.  These mappings should be both continuous and effective, meaning the intentions of an agent should be taken into consideration in any affective augmentations.  How can an embedded intelligence controlling the augmentation infer the expectations of an agent in order to create realistic and perceivable augmented realities?  The current sub-chapter begins to answer this question by reviewing the literature in two essential mechanisms of visual perception: attention and object representation.  Beginning with an overview of eye-movements, the review continues to discuss two well-studied phenomena indicative of the architecture of early visual representation: Gist and Change Blindness.  Finally, the review concludes in a discussion on building a computational model of visual perception based on the presented literature.  
\end{abstract}

\section{Introduction}

%\begin{quotation}
%``Most of us are brought up to feel that what we see out in front of us is something that lies beyond our eyes—out here. That the colors and the shapes that you see in this room are out there. In fact, that is not so. In fact, all that you see is a state of affairs inside your head. All these colors, all these lights, are conditions of the optical nervous system. There are, outside the eyes, quanta, electronic phenomena, vibrations, but these are not light, they are not colors until they are translated into states of the human nervous system. So if you want to know how the inside of your head feels, open your eyes and look. That is how the inside of your head feels.'' Alan Watts
%\end{quotation}

In building an augmented reality, perceptual mappings between an agent and the augmented content of the environment they perceive should be both continuous and effective, meaning the intentions of an agent should be taken into consideration in any affective augmentations.  How can an embedded intelligence controlling the augmentation infer the expectations of an agent in creating realistic and perceivable augmented realities?  The current sub-chapter begins to answer this question by presenting an overview of literature on visual perception and its possible representations with the final aim of motivating the basis for a computational model.

Beginning with an overview of attention and eye-movements, the sub-chapter continues to discuss a number of presiding architectures for visual perception built as a result of studies in Gist and Change Blindness.  Finally, the review concludes in a discussion on developing a computational model of early visual perception using the presented evidence.  

\section{Attention}

Our experience of the world is a rich, continuous, and fully-detailed illusion.  Yet, our eyes rapidly move an average of 3-5 times a second, completely disrupting the continuity of light entering our eyes.  Visual acuity limitations mean that our eyes require rapid ballistic movements of the eye taking all of 30 ms (a \textit{saccade}) to project the light from the particular point of a visual scene we are interested in onto a 2-degree area of the retina with the highest spatial resolution (the \textit{fovea}).  Going away from the fovea (the \textit{parafovea}), resolution for spatial detail drops logarithmically, while resolution for motion detail increases, a relationship due to the distribution of photo-receptive cells in the eye combined with the lens of the eye itself.  We cannot encode with high spatial detail an entire visual scene, as a camera with a small aperture may be able to do, and require saccades (and head-movements) to move our eyes to stabilize (a \textit{fixation}) of our eyes to the region of interest, a process lasting on average 330 ms.  During this time, it is thought that encoding into memory occurs as well as planning of the next eye-movement. 

The earliest studies in eye-movement behavior \cite{Buswell1935,Yarbus1967} describe two main influences of a viewer's attention to a visual scene: (1) influences dependent on mental states which focus attention towards contextually and cognitively relevant aspects of the world (\textit{endogenous}), and (2) influences dependent on involuntary capture of attention from the external environment (\textit{exogenous}).  As exogenous factors are involuntary, one would expect to find the behavior influenced by these factors to be highly consistent across viewers.  In contrast, as endogenous influences are dependent on cognitive factors resulting from emotion, memory, language, task, and previous experiences, the relation of a scene and one's endogenous influences on the scene are much less consistent across viewers.

\subsection{Exogenous Influences on Attention}

In seminal work investigating the speed of visual perception using Gestalt primitives, Sziklai demonstrated the human visual system exhibits an attentional bottleneck of 40 bits per second on selected information, suggesting our visual systems require a simplified representation from the many megabytes per second of information coming from exogenous visual information \cite{Sziklai1956,Merrill1968}.  Much research investigating exogenous influences on static visual scenes therefore describe a simplified representation of attentional control known as a \textit{bottom-up} model \cite{Koch1985,Itti1998,Wolfe1989,Itti2001}.  Such models are built around theories of feature-integration \cite{Treisman1980} and are further supported by physiological evidence of the receptive fields and visual architecture of the visual cortex of cats \cite{Hubel1962}.  To discover the attentional biases for portions of a scene (\textit{saliency}), bottom-up models recompose a full resolution image using filter banks tuned to multiple frequency orientations and scales corresponding to pre-attentive visual features also found in early visual cortex such as luminance, oriented edges, and color contrasts.  Saliency is then computed as a weighted linear summation (\textit{integration}) of the resulting ``feature maps'' formed of different scales. 

It is thought that basic feature levels of models of integration are modulated by ''top-down'' influences \cite{Itti2001} such as the current ongoing task \cite{Yarbus1967,Smith2011a} and the context of a scene in order to reduce processing load \cite{Henderson2003,Torralba2006}.  Though, the level at which top-down influences may affect processing is still open to debate.  Further, though these modulations are often described as top-down influences, such a term should not be confused with endogenous influences, as much research has shown that memory, context, and other endogenous factors affect early visual processing \cite{Tatler2011} which would correlate with initial feature stages thought to be unaffected in a bottom-up model.  %Further, gist-based models (presented later) separating ``conceptual'' and ``perceptual'' influences may similarly be looked at in terms of ``bottom-up'' and ``top-down'' influences, or even ``exogenous'' and ``endogenous'', though these terms all vary slightly in their use and meaning and should be used with caution.  
  	
\subsection{Endogenous Influences on Attention}
\label{sec:endogenous-influences}

In a seminal study on how task affects eye-movements during static scene viewing, \cite{Yarbus1967} tracked the eye-movements of participants viewing a painting entitled, ``An Unexpected Visitor.''  His study showed that when participants viewed the painting and were given a task such as to determine the ages of the people in the painting, they looked more at the faces of each person.  When asked to determine what they were wearing, their eye-movements strayed away from faces, and looked more towards the clothing of people.  Yarbus further describes 7 different tasks and shows how the eye-movements of each participant reflects the information required for processing the task at hand.  It is thought that task, therefore, is an endogenous influence.

In a similar study on dynamic scene viewing, Smith studied task-based effects on viewers' eye-movements looking at unedited videos of natural scenes from a camera mounted on a tripod \cite{Smith2011a}.  Participants were natives to the city of Edinburgh and viewed a variety of indoor and outdoor scenes from the city.  The study revealed that during free-viewing, i.e. not given any task other than to look at the video, participants looked at mostly moving objects such as people moving across the frame or cars.  However, when given the task to identify the location of the presented scene, participants had to concentrate their gaze towards the elements of a scene depicting landmarks such as buildings, signs, and trees and showed a remarkable ability to distract away from moving objects.  After viewers pressed a button indicating recognition of the location, their viewing behavior reverted to resembling the free-viewing task, fixating on moving objects such as people and cars again.  The study re-asserts the findings of Yarbus, though for a dynamic time-course.  Further, it also provides evidence of default viewing conditions during the time-course of viewing, as participants were able to ``return'' to the free-viewing task after having finished the task of recognizing the location of the scene.  

\section{Gist}
\label{sec:gist}

The ability to classify scenes with rapid pre-attentive processing lasting only 45-135 ms (\textit{Gist}) \cite{Potter1969,Biederman1974,Potter1976,Schyns1994,Henderson1999} suggests that the general shape and structure of a scene leading one to infer its context are defined by either volumetric forms (\textit{geons}) \cite{Biederman1987}, spatial arrangement of blobs defined by contrasts in luminance or color \cite{Schyns1994,Oliva1997} or by using a scene's spatial frequency content \cite{Oliva2001,Oliva2005}.  A scene's spatial frequency content can be described by oriented band-pass filters: at a low spatial frequency, this content resembles broad edges and the layout and orientations of a scene's largest similarly textured regions, whereas at a high-spatial frequency, the response of the sharpest edges and their directions are encoded.  

Endogenous influences on subsequent processing of gist seem to influence the spectral scale at which gist is selected \cite{Schyns1994,Oliva1997}.  Schyns and Oliva describe an experiment where a low-spatial frequency (\textit{LSF}) and a high spatial frequency (\textit{HSF}) image are created for two separate pairs of images.  Creating two new images by combining the LSF of one image and the HSF of the other, and vice-versa, they investigate the scale space of gist recognition with and without a verbal cue to indicate what type of scene will follow (\textit{priming}).  Without priming, subjects are able to recognize the scene described by the LSF content of an image given 45 ms of presentation time, and the HSF one within 135 ms.  As well, subjects are unaware of the content in the other scale space (i.e. shown an image with LSF and HSF content for 45 ms, the participants are unaware of there being separate HSF content).  However, being primed with either the LSF or HSF content of the scene, subjects report perceiving the given cue instead.  Thus, while gist is thought to be pre-attentive, i.e. before the timescale of acts of selective attention, such research suggests either that (1) the scale at which the early representation of gist operates at is affected by task-demands (i.e. only one scale of gist is encoded for pre-attentively), or (2), attention and further encoding into memory is dependent on endogenous influences on scale selection, (i.e. gist may be encoded at multiple scales, but only the scale selected by attentional machinery is encoded into memory).  Though not all scales are necessary for determining a scene's content when given prior cues (textit{priming}), the neurobiology of early visual cortex gives scope for encoding of multiple visual scales.  It thus seems possible to assume (2) is a more likely model for the interaction of gist and attentional machinery.

\section{Change and Inattentional Blindness}

Research over the last century demonstrating the failure to report large changes in the visual world (\textit{change blindness}) as well as the failure to report unexpected visible changes due to task requiring attention elsewhere (\textit{inattentional blindness}) \cite{Simons1999,Rensink2000,Rensink2001,Hollingworth2001a} have shown that our visual systems are unaware of changes in visual world outside of the point of fixation.  Simons and Chabris demonstrated ''Inattentional Blindness'' by composing a video of two basketball teams dressed in white and black passing a ball to each other \cite{Simons1999}.  Participants were asked to count the number of passes that the white team makes.  During the course of the video, a person wearing a gorilla suit walks across the frame of the camera, unnoticed by 75\% of participants.  The phenomena of ``Change Blindness'' was demonstrated in a real-world psychology experiment \cite{Simons1998} where participants arrived at a kiosk to fill in a consent form and hand the completed form to a man behind the counter.  The man ducks behind the counter as to pretend to file the paper, while a different man comes up from behind the counter, again unnoticed by a majority of the participants.  

Failing to detect changes outside of the point of fixation suggests that any peripheral representation of a scene would likely not encode details of object specific features such as color, motion, or orientation gratings.  Rather, our visual machinery integrates the detailed aspects of objects across eye-movements, retaining that information as a perceived representation of the visual world.  What form, and to what detail this representation encodes is still an open question.  Rensink takes this evidence in developing a theory of coherence, proposing that object representation depends on focal attention.  For objects outside of the point of fixation, Rensink proposes we encode volatile units of ``proto-objects'' \cite{Rensink2000,Rensink2001}.  Proto-objects are argued to be amorphous and blob-like in nature, representational-less and concept-less lasting only a few hundred milliseconds.  It is further argued that attention operates on groupings of proto-objects rather than at the earlier feature levels making it the highest level of early vision, and the earliest operands of selective attention.  Rensink also hypothesizes that proto-objects may explain non-attentive processes capable of recognizing the abstract meaning of a scene and the spatial layout of the scene \cite{Rensink2002}.  In relation to perceptual influences, implicit behavioral measures suggest that grouping processes can also occur for task-irrelevant visual stimuli, i.e., for stimuli that has not been attended to by a fixation, further supporting theories of proto-object formation \cite{Lamy2006}.

\section{Discussion}

Research in change blindness has indicated that though we experience a rich, detailed visual world, we do not use such rich details in building a stable representation \cite{Simons1997}.  Rensink argues that object representation requires focal attention.  However, in considering an architecture of visual perception, what is the cause of producing focal attention?  The literature presented here suggests that there is either an endogenous explanation or exogenous one.  For example, I may focus on a cup, but not build the representation of the fingerprints on the cups as I was not intending to look at this particular scale.  In this case, the endogenous influence of perceiving the object representation of fingerprints on the cup was necessary for building such a representation, even though focal attention will have brought my eyes to the cup.  It may be that my task of drinking from the cup saw the cup as what it afforded: a drink.  In a free-viewing task, if such a thing exists, it may be more likely that an exogenous influence such as the mis-representation of the cup will provoke more detailed representations and cause additional focal attention to the cup.  Thus, it may be the case that focal attention is necessary for explaining an object, however, it seems it is not sufficient and the cause of focal attention should still be considered.  

When considering evidence for gist in relation to Rensink's theory of coherence, it seems viable to consider proto-objects as the same representation that gist may use \cite{Rensink2002}.  Though Schyns and Oliva argue for using oriented banded filters, it is not unlikely that collections of blob-like entities which necessarily also respond to the scale of the proto-object could provide a cue for spatial layout.   However, when considering evidence in rapid determination of the meaning of scenes, Schyns and Oliva demonstrated that early processing of a scene could be re-organized based on prior experiences \cite{Schyns1994,Oliva1997}.  Thus, it is not clear from their research alone whether the pre-conceptual representation itself can be changed, or if only the attentional machinery acting on a set of possible representations has changed.  The latter effect would entail a sort of conceptual prior on a scene, suggesting the organization of a scenes early representation remains untouched.

Pylyshyn theorizes that the understanding of a concept is not all that is required for visual experience: 
\begin{quotation}
''Vision suited for the control of action will have to provide something more than a system that constructs a conceptual representation from visual stimuli; it will also need to provide a special kind of direct (preconceptual, unmediated) connection between elements of a visual representation and certain elements in the world. Like natural language demonstratives (such as 'this' or 'that') this direct connection allows entities to be referred to without being categorized or conceptualized. \cite{Pylyshyn2001}''
\end{quotation}  
The preconceptual connections Pylyshyn describes are easily described by the pre-attentive proto-objects Rensink also describes \cite{Rensink2000,Rensink2001}.  What is interesting in Pylyshyn's theory is the notion that this pre-conceptual representation does not need to be categorized or conceptualized in order to be referred to.  In other words, the categorization which Pylyshyn theorizes of is part of the attentional machinery which refers to proto-objects, rather than an explicit property of the proto-object themselves.  According to Pylyshyn's theory, proto-objects of a visual scene are then described by one particular fate, and attentional mechanisms can only select from the set of possible proto-objects, rather than influence their definition. 

Considering both the implicit, unmediated representation and the attentional and contextual mechanisms, at least two critical layers should be built into any computational model based on the evidence presented here:  (1), a pre-conceptual representation which takes into account different possible spatial configurations, composed of either band-passed edge-oriented filters, geons, or proto-objects, where this representation is affected by a logarithmic filter around the point of fixation based on the evidence of response properties of photo-receptors; (2), an attentional and contextual influence supported by the ongoing experiences of the subject such that parafoveal information becomes unstable without ongoing attention and is only inferred by through the context of the scene.  The intentions of an agent within this model are still not well-understood, as the variety of possible endogenous influences that may be possible are too great.  

Similar computational models have been developed to explain visual perception machinery \cite{Walther2006,Orabona2007a}, however they each suffer from a number of problems: (1) they lack the inclusion of the evidence of the response properties of photoreceptors in the retina as there is no indication of the current or ongoing attention within the visual scene; (2) they infer context based on solely a static image whereas the real-world is dynamic; and (3), they cannot distinguish groupings of proto-objects and instead create discrete maps which are thresholded as attention or saliency maps.  Furthermore, the interest in the previously cited models of visual perception is in predicting attention towards a scene, rather than allowing an agent in the world to explicitly define this.  In such a case, these models are unsuitable for applications in augmented or virtual reality where the agent already provides attention within a scene.  

