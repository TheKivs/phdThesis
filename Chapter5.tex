\chapter{Visual Scene Analysis}
\minitoc

\begin{abstract}
Building convincing augmented realities requires creating perceptual mappings between an agent and the augmented content of the environment they perceive.  These mappings should be both continuous and effective, meaning the intentions of an agent should be taken into consideration in any affective augmentations.  How can an embedded intelligence controlling the augmentation infer the expectations of an agent in order to create realistic and perceivable augmented realities?  The current sub-chapter begins to answer this question by reviewing the literature in two essential mechanisms of visual perception: attention and object representation.  Beginning with an overview of eye-movements, the review continues to discuss two well-studied phenomena indicative of the architecture of early visual representation: Gist and Change Blindness.  Finally, the review concludes in a discussion on building a computational model of visual perception based on the presented literature.  
\end{abstract}
