\chapter{Introduction}
\label{chap:intro}
\minitoc

\setlength{\epigraphwidth}{0.6\textwidth}
\epigraph{\flushright{These fragments I have shored against my ruins.}}%
{-- \textsc{T.S. Eliot}, excerpt from \textit{The Waste Land}}%

\section{Motivation}

Collage is an arts-practice which appropriates cultural fragments for its raw materials.  Depending on its medium, it juxtaposes, often chaotically, fragments such as photographs, text, or clips of sound, removing them from their original context.  Collage is capable of producing new interpretations which the original fragments alone could not have provided.  It is inherently a process that invokes new ways of seeing creating a meaning-making process between the artist producing the chaos and the audience that must unify its cut-up percepts into order.  More than an artistic technique, it has also been described as a ``philosophical attitude'' that can be applied to virtually any medium \cite{McLeod2011}.  

%As collage-based practices have progressed over the last 100 years, media too has evolved.  As a result, the tools and technologies available for manipulating and accessing media have altered the practice.  

This thesis attempts to build a computational framework for generating a particular type of collage I call a scene synthesis.  Scene synthesis places emphasis on using psychologically-motivated representations for the units of collage.  Across numerous literature these representations are known as clusters, gestalts, geons, proto-objects, or streams.  Computationally, we will discover that these representations are defined by ``temporally coherent'' information in space-time stimuli.  The interest in defining collage by these units is to mimic early-processes of perception presenting a dialogue into how perception may operate.  

This thesis will discuss how scene synthesis has been used for a range of applications. The framework for scene synthesis presented here is rooted in building interactive dynamic collages where a participant can experience a collage being generated in real-time.  In one incarnation available as an iOS app, ``Memory Mosaic'', a participant experiences a real-time collage of sound clips.  The sound clips are aggregated in real-time from the current environment and are meant to represent fragmented sonic memories.  The resulting experience continually associates the ongoing sonic world to its memories creating a real-time sonic collage.  In an analogous fashion, the visual equivalent is built in another iOS application called ``PhotoSynthesizer''.  A user can select a range of source photos and use them to synthesize any target photo.  The synthesis is drawn as a painting, starting from the background and eventually reaching the finer details of the foreground layers.  This framework is also applied for producing artistic stylizations of existing images or videos and presented at the ACM Symposium on Applied Perception.  Finally two outputs explore scene synthesis for both audio and visual modalities: (1) ``Augmented Reality Hallucinations'', exhibited during the London Design Festival at the V&A Digital Design Weekend, which uses an augmented reality headset originally built for immersive gaming environments delivering the visual experience and headphones to produce the sonic experience; and (2), ``YouTube Smash Up'', which synthesizes the number 1 video on YouTube of the week using the top 2 - 10 videos of the week.   Through these applications and a number of experiments along the way, we will discover when scene synthesis can work and when it cannot.

\section{Background}

This section situates the thesis within a lineage of collage-based practices starting with seminal developments in collage, montage, cut-up, and musique concr\`ete, before moving to an overview of modern developments in technology that radically changed collage practice in terms of its practice, publishing, distribution, and even legality.  Finally, computational methods mostly rooted in application-oriented rather than arts-oriented practices are discussed.  

\subsection{Early 20th Century Collage-based Practices}
Though early practices of collage such as the invention of paper in China around 200 BC, calligraphers in Japan in the 10th century, the 15th and 16th century practices of adding gold leafs or other gemstones to canvases, and Giuseppe Arcimboldo's 16th century experiments with portraits composed of fruits and vegetables, it was not until the first-half of the 20th century that an explosion of collage practices spanning visual, textual, and sonic mediums was exhibited.

In 1912, Pablo Picasso transformed a still-life by gluing an oilcloth to the canvas in \textit{Still Live with Chair Caning}.  Along with Georges Braque, they experimented with gluing visual fragments of culture represented by stamps, newspaper clippings, and photos to their canvases.  Shortly after in the 1920's, collage-based practice would define a major tenant of the Berlin-Dada scene as seen in the work of George Grosz, John Heartfield, and Kurt Schwitters who would often work entirely with found objects meant to serve as representations of the city (e.g. \textit{Irgendsowas}, 1922).  Around the same time, collage found its way to purely cut-up photographic material, a technique known as photomontage, which can be seen in the work of Max Ernst's \textit{Murdering Airplane} (1920) and Hannah H\"och's \textit{Pretty Maiden} (1920).

With the surge of developments of collage-based practices occurring on the canvas, it is no surprise that collage would find its way to text mediums as well.  At a Dadaist rally in the 1920's, Tristan Tzara performed a poem by taking cut-up fragments of text-based media such as newspapers or brochures out of a hat and reading them aloud (eventually leading to a riot destroying the theatre on location and the expulsion of Tzara from the movement by Andre Breton).  The ``cut-up technique'', as it is now known, can be found in some of modern literatures greatest works, such as T. S. Eliot's \textit{The Waste Land} and James Joyce's \textit{Ulysses}.  Cut-up is later rediscovered by Brion Gysin who supposedly accidentally rediscovered the technique and consequently shared the technique to William Burroughs in the 1950's.  Burroughs made extensive use of cut-up in his writings, most notably in \textit{Naked Lunch} where each chapter was to be read in any order, and in his collaboration with Gysin, \textit{The Third Mind}.  The technique would also eventually find its way into mainstream music in the lyrics of David Bowie, Kurt Cobain, and Radiohead.  It is worth noting that with the exception of Tzara's performance of cut-up, these examples of text-based collage do not include the usual distinguishing markers of collage as the boundaries of the source material may not be immediately obvious \cite{McLeod2011}.

The Surrealists also shared the collage-based practice of forming new meanings through the collection of fragmented material.  Though while not strictly collage, Andre Breton writes of a popular parlor game amongst Surrealists in the 1920's called ``Exquisite Corpse'' which had participants collectively assemble text or images often using simple rules such as, ``adjective then noun''.  Breton later describes the game: ``they bore the mark of something which could not be created by one brain alone...fully liberating the mind's metaphorical activity'' \cite{BretonRemembers}.    

Collage practices did not stop with static media, however.  In 1925, the Russian film director Sergei Eisenstein demonstrated the power of film montage in \textit{Battleship Potemkin} as he juxtaposed image sequences such as a crowd's flight down a staircase with the image sequence of a baby carriage for 7 minutes, creating viscerally new experiences and emotions than either sequence alone could have.  While not strictly collage as it had been, the technique of producing new meanings from the collection of individual fragments was once again shared.  

By the end of the 1940's, radiophonic art, or the practice of producing sound for radio broadcast, had been well established.  Words, music, and noises were combined to produce radio productions of literary stories and news broadcasts.  It is no surprise then that in one studio in Paris, France, Pierre Schaffer was also experimenting with splicing and recombining magnetic tape recordings of sound in a practice later called musique concr\`ete \cite{}.   Say something more about how it was later theorized in ``The Guide to the Sonic Object'' and how the practice differed from radiophonic art \ref{}.

\subsection{Modern Collage-based Practices}
Many developments in technology significantly altered collage-based practices, allowing it to explore entirely new processes for composition, new mediums and new methods of presentations, and new audiences. 

For instance, the Xerox Corporations development of the photocopier led to a surge of do-it-yourself small-publishing houses.  As well, it led to the development of xerox-based collage artists, most notably featured in a bi-monthly xerox-printed zine called \textit{PhotoStatic} which had a peak circulation of 750 copies in the 1980's finally ending with its 41st issue in January 1993 \cite{McLeod2011}.  

As well, Polaroid's introduction of instant film allowed photographers to instantly see their developed film without the need for mixing chemicals.  David Hockney explored their use creating patchwork collages (for instance, using as many as 63 Polaroids in \textit{Still Life Blue Guitar}, 1982) which he called ``joiners''.  The individual patches comprised of individual photos that were taken at different times effectively creating different exposures, lighting conditions, and vantage points.  Hockney's interest in the joiner's were intimately tied to human perception, stating that they made ``things closer to the truth of the way we see things'' \cite{Joyce1988}, i.e. perception constructed through the fragments of many perspectives.

1963 saw pivotal developments for the field of interaction with computers and in particular with computer graphics: Douglas Englebart's computer mouse and Ivan Sutherland's sketchpad (an extension of Robert Everett's light pen for the TX-2).  Both allowed a user to isolate and drag pixels on a screen, moving them from one data-space to another.  DJ Spooky remarks on the transformations these and other pioneers in human-computer interfacing technologies would have on the arts: ``Douglas Engelbert and Ivan Sutherland pioneered graphical user interfaces...but what they accomplished was even more profound than that, their work let us move into the screen world itself'' (quoted in \ref{Jaeger2006}).  

It is in the 'screen world' that a wealth of new technologies would be developed in the coming years to allow the manipulation of data representing media.  As a result, perhaps the most significant deviation from early collage practice comes from the advent of digital media.  As media was no longer contained to the physical world, but a virtual screen-based one, where only software could manipulate the media, collage practice was intimately tied to the capabilities of software.  Software for page layout, illustration, graphics, and photo/sound/ and video editing have afforded practitioners with faster and more sophisticated methods of collage.  For instance, Adobe Photoshop supports the automatic parsing of an image into object regions that can be individually manipulated enabling artists to finely segment and compose visual media.  For video, Adobe After Effects can similarly segment objects and track features across frames, making dynamic visual collages much easier to do.  

Various other software such as Director, Flash, VJamm, VDMX, Quartz Composer, Jitter, Resolume Avenue, VVVV, Modul8, or Processing the created greater possibilities for video-based media than standard non-linear editors such as Adobe Premiere or Apple Final Cut Pro could offer \cite{Jaeger2006}.  These tools opened video-based practices to real-time and performative practices creating Visual Music and VJ cultures.  Many artists even create their own software, releasing them to the public such as Matt Black of Coldcut, who created VJamm, Netochka Nezvanova who wrote Nato.0+55, or Miller Puckette, author of pd \cite{Jaeger2006}.  

Collage-based video practitioners in the 1990's would often use clips of appropriated television footage such as the news footage of the motorcade leading to JFK's assasination in Steinski's ``The Motorcade Sped On'', George H. W. Bush's news broadcasts on the Gulf War in EBN's ``We Will Rock You'', or Coldcut's use of nature videos.  Although, these examples, as well as more recent artists such as Eclectic Method, are more like the early use of montage of film, as their spatial cut-up, i.e. across a single frame, generally remain intact with the exception of the use of ``blending'', a technique with takes 2 or more images and blends the two together using a variety of functions such as: add, multiply, subtract, difference, and so on.

Perhaps the most sophisticated use of visual collage-practice comes from Czech surrealist animator Jan Svankmajer, who combines stop-motion and collage-based techniques within his animations to create intricately detailed and surreal dynamic portraits and landscapes.  For example, in ``Dimensions of Dialogue'', Svankmajer creates the image of two heads composed of a variety of coarsely fragmented objects in the style of Giuseppe Arcimboldo.  Their ``dialogue'' ensues which effectively entails the one head eating the other one, chewing the head momentarily, then regurgitating it as a finer representation of objects.  The process continues in turns with one head eating the other until they resemble smooth clay-like sculpted heads.  

For sound, collage-baed practices exploded with the birth of digital sampler hardware.  Similar to the practice of collage or montage, sampling refers to taking portions of existing media and using it within a performance or composition.  Digital sampling refers to sampling after an analog-to-digital process, where physical vibrations of sound are digitally sampled at equally spaced intervals, creating a discretely sampled representation of the continuous real-world phenomena.  This discrete sampling can easily be encoded by 1's and 0's in computer storage, and easily decoded back to the physical world, via an digital-to-analog process, such as through a speaker.  

Early digital samplers such as the Fairlight CMI to more recent non-linear editors such as Logic and Ableton Live have made multi-track and cut-and-paste operations trivial to accomplish, while visualizing sound waves has made finding relevant parts of an audio file relatively easier than listening to an entire tape reel.  Early adopters of digital sampler technology include Herbie Hancock and Public Enemy.  Founders of Dub, Lee ``Scratch'' Perry and King Tubby, also made use of existing recorded material.  Though not strictly sampling, they created the famous Dub sound by infinitely collaging the same sound, creating intense reverberations that echoed with greater magnitude on each bounce until the sound was cut.  

In 1987, the KLF produced ``What the Fuck Is Going On?'', which made extensive use of samples from The Monkees, The Beatles, Dave Brubeck, Led Zeppelin, Whitney Houston, and ABBA, amongst many others, citing on the album liner notes that the samples had been freed ``from all copyright restrictions''.  Despite their claims, their independent release was ordered to be destroyed by the Mechanical-Copyright Protection Society, leading to a re-release of the album with periods of protracted silence in place of the unauthorized samples.  They also released a guide including a detailed construction of how to reproduce the sound of the album, including the hardware used: an Apple II computer, a Greengate DS3 digital sampler peripheral card, and a Roland TR-808 drum machine.  

Two years later in 1989, John Oswald released an amalgamation of sampled music in his album, ``Plunderphonics'', including a visual reference to Michael Jackson's ``Bad'' on its album, which featured a derivative image of Jackson's original album cover for ``Bad'' edited to make it look like Jackson was a naked woman wearing a leather coat.  On the album, a song by the name, ``Dab''  was collaged to create the essence of a Michael Jackson track, using samples from Jackson's ``Bad''.  Oswald describes his process of sampling as using ``plunderphones'', describing them as, ``a recognizable sonic quote, using the actual sound of something familiar which has already been recorded'', satisfying the essence of being a plunderphone ``as long as you can reasonably recognize the source'' \cite{OswaldInterviews}.

In his writings available online, he further describes his motivations: ``'Plunderphonics' is a term I've coined to cover the counter-covert world of converted sound and retrofitted music where collective melodic memories of the familiar are minced and rehabilitated to a new life'' \cite{Steenhuisen2005}.  Unfortunately for Oswald, the Canadian Recording Industry Association ordered him to cease-and-desist production and to destroy all remaining copies.   Oswald's collage-based practice also extended to text, taking cut-up fragments from existing authors without citation, even including un-cited quotes to his own previous text ``Plunderphonics, or Audio Piracy as a Compositional Prerogative'' in ``Creatigality'' and ``Bettered by the Borrower'' \cite{Tholl}.  

Perhaps the most pervasive and popular use of sampling, or creative plagiarism as Kembrew McLeod cites it \cite{McLeod2011}, however, came in the form of Hip-Hop music.  Public Enemy's song, ``Caught, Can I Get a Witness?'', released in 1988, remarks on the practice of digital sampling: 
\begin{verse}
Caught, now in court 'cause I stole a beat\\
This is a sampling sport
\end{verse}

Negativland, 1991

Stop Motion -> Jan Svankmajer -> Pink Freud: https://www.youtube.com/watch?v=5NPiOjFVQRc

Computer Graphics -> boundaries of collage become less clear: https://www.youtube.com/watch?v=fw3XyOyl47Q

Computer collage will proliferate so long as ``cut'' and ``paste'' remain essential operations to collage, described by Taylor in ``Collage''.  Joseph Nechvatal, computer-robotic assisted paintings; 

Space of collage: Andruid Kerne's Collage Machine; I/O/D's WebStalker; Mark Napier's Riot; 

Vinyl -> DJ'ing

VJ'ing (VDMX, Quartz Composer, Jitter, Resolume Avenue; VVVV; Modul8). EBN; Coldcut in the 90's who used nature videos such as monkeys jumping, beetles tapping, and plants opening and synchronized their motions to sound samples using VJamm. Eclectic Method using Algoriddim's VJay iPad app.

Scratch Video of British 1980's 

Situationists detournment

William Burrough's Electronic Revolution

Photo-silk-screen

Photo-litho

Real-time video 

Computer Graphics

% Corel Draw, Director, Flash

\subsection{Computational Collage-based Practices}

Developments in machine vision/listening led to new tools/applications;

Aaron Hertzmann; SIGGRAPH community

Diemo Schwarz's CataRT; Nick Collins's BBCut and klippAV

Ben Bogart's Dreaming Machine

%\subsection{Media Libraries}
%As well, the amount of media has grown, and its access has only become easier with the advent of online archives such as YouTube, Flickr, Soundcloud, Freesound, just to name a few. 
% Shifting dialogue from war, communism, capitalism, to property, ownership, copyright
% in 1979 written about by douglas kahn, sad news by rob summers: https://www.youtube.com/watch?v=H4BmS9krF1A
%\subsection{Information Retrieval}
%\subsection{Information Visualization/Auralization}
%As collage also works with visualizing large amounts of existing data, the work also shares motivations with information visualization and information auralization, where the data are presented in order to tell a story.  
%\subsection{Compression/Cryptography}
%In another vein, the work also shares motivations with encoding and decoding, or compression.  The purpose in compression is to take an existing dataset and reduce it down to ``perceptually'' similar information, while removing extraneous information.  
%In a related note, this process can also be used for encryption/decryption, where the same process is used to embed meaningful information that can only be deciphered through arcane processes, such as a table look-up or transformation of the data.  

\section{Goals}

This thesis investigates a computational method for scene synthesis, an automated collage generation where the units of the collage are based on psychologically-motivated representations.  

%A set of parameters effect the two major modular components of the algorithm: attention modeling and representation.   Interaction is focused on the selection of material used in the collage, how the source content is parsed, stored, and for dynamic media, how the collage should be composited over time.  This system affords entirely new experiences around collage, such as the ability to aggregate source content in real-time or experience the collage through an augmented reality headset that mosaics the worlds as it is experienced.  

Encode only parts of a scene that are likely to attract attention... motivate attentional model for dynamic content... 

Develop representation of audio and visual corpus that affords simple interaction to produce different styles...

Fragments of a collage require precarious balance between what is identifiable, i.e. how discernible it is as the original source, and what can be composited, i.e. how it can fit within the greater context. 

Difficulty of evaluating something subjective; not impossible; can measure performance of speed; art critics...

\section{Overview}

Basics:
	- Attention literature
	- Representation literature

Auditory Scene Analysis

Auditory Scene Synthesis
	- Memory Mosaic app
	- Daphne Oram Browser
	- Infected Puppets (exhibited in India at the Bangalore Artist Residency and developed in collaboration with 12 students at Srishti School of Art, Technology, and Design's Center for Experimental Media Art)
	
Visual Scene Analysis

Visual Scene Synthesis
	- Photo Synthesizer app; Have contact from an artist, Frieso Boning (The Winnipeg Trash Museum) who used it in their arts practice to create some very nice photos...
	- Artistic Stylization of Image/Video

AudioVisual Scene Synthesis
	- YouTube Smash Up; Copyright Issues; Validation through Copyright Infringements
	- Augmented Reality Hallucinations; exhibited at V&A; Feedback from 21 participants

Conclusion








