\chapter{Introduction}
\label{chap:intro}
\minitoc

\section{Motivation}
The first-half of the 20th century saw an explosion of collage practices spanning visual, textual, and sonic mediums.  In the early 20th-century in Paris, Picasso and Braque were experimenting with gluing visual fragments of culture represented by stamps, newspaper clippings, and photos onto the canvas of their paintings.  At a Dadaist rally in the 1920's, Tristan Tzara created a poem by taking cut-up fragments of existing texts such as newspapers or brochures out of a hat and verbalizing them\footnote{leading to a riot destroying the theatre on location and the expulsion of Tzara from the movement by Andre Breton.}.  At about the same time in 1925, the Russian film director Sergei Eisenstein demonstrated the power of film montage in \textit{Battleship Potemkin} as he juxtaposed image sequences such as a crowds flight down a staircase with the image sequence of a baby carriage for 7 minutes, creating viscerally new experiences and emotions than either sequence alone could have.  And by the end of 1940's, Pierre Schaffer and Pierre Henry were experimenting with splicing and recombining magnetic tape recordings of sound in a practice they called musique concrete.  

Since then, collage practices have only evolved, becoming more ubiquitous and easier to do.  As media has moved into the digital domain, the possibilities for collage have allowed for large corpora of media that would have taken years or even lifetimes to edit if done through their analog counterparts.  However, the current state-of-art of software enabling digital collage practice has not allowed for for more intelligent parsing of content.  For instance, if I would like to collage an audio track or image, I still have to know which file and what part of the file I require for my collage.  

This thesis investigates a computationally creative model for automating collage practice, allowing for transformational forms and experiences of collage.  

\section{Goals}

\section{Overview}