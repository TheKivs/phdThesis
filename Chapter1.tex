\chapter{Introduction}
\label{chap:intro}
\minitoc

\epigraph{
These fragments I have shored against my ruins. \\
-- T.S. Eliot, \textit{The Waste Land}
}

\section{Motivation}

Collage is a practice that uses existing fragments of media to create new meanings.  The juxtaposition of  appropriated cultural fragments such as words, photos, or sound clips, the situation of the content's reuse, and the collage's overall composition produces new meanings that the original fragments alone could not have provided.  Indeed, more than an artistic technique, it has also been described as a ``philosophical attitude'' that can be applied to virtually any medium.

As collage practice has progressed over the last 100 years, media too has evolved, meaning the tools and technologies available for manipulating and accessing media have altered the practice.  This thesis looks at how developments in machine listening and machine vision can create new representations for the units of collage and automate the generation of collage in order to build interactive collage experiences that a participant can experience in real-time.  In one incarnation, ``Memory Mosaic'', a participant experiences a real-time collage of aggregated memories, associating the ongoing audiovisual world to fragments of previous experiences.  In another, the collage is experienced through an augmented reality headset, where the vision and sound of the environment is presented through a headset originally built for immersive gaming environments.  The resulting experience, ``Augmented Reality Hallucinations'', captures the experiences of the viewer through their attention to the world, translating them into a collage of fragmented memories.  

\section{Background}

\subsection{Early Collage}
The first-half of the 20th century saw an explosion of collage practices spanning visual, textual, and sonic mediums.  In the early 20th-century, Pablo Picasso and Georges Braque were experimenting with gluing visual paper fragments of culture represented by stamps, newspaper clippings, and photos onto the canvas of their paintings.  The practice was later extended to wood by Kurt Schwitters in the 1920's, and to purely cut-up photographic material in the 1950's, a technique known as photomontage.

At a Dadaist rally in the 1920's, Tristan Tzara created a poem by taking cut-up fragments of text-based media such as newspapers or brochures out of a hat and verbalizing them eventually leading to a riot destroying the theatre on location and the expulsion of Tzara from the movement by Andre Breton.  The technique, also known as ``cut-up technique'', formed the basis of T. S. Eliot's \textit{The Waste Land} and was later rediscovered by Brion Gysin who had shared the technique to William Burroughs in the 1950's.  Burroughs made extensive use of the technique in his writings, most notably in \textit{Naked Lunch}.  The technique would also eventually find its way into mainstream music in the lyrics of David Bowie, Kurt Cobain, and Radiohead.  

Collage did not stop with static media, however.  In 1925, the Russian film director Sergei Eisenstein demonstrated the power of film montage in \textit{Battleship Potemkin} as he juxtaposed image sequences such as a crowds flight down a staircase with the image sequence of a baby carriage for 7 minutes, creating viscerally new experiences and emotions than either sequence alone could have.  While not strictly collage as it had been, the notion of producing new meanings from the collection of individual fragments was certainly shared.  

By the end of the 1940's, radiophonic art, or the practice of producing sound for radio broadcast, had been well established.  Words, music, and noises were combined to produce radio productions of literary stories and news broadcasts.  It is no surprise then that in one studio in Paris, France, Pierre Schaffer was also experimenting with splicing and recombining magnetic tape recordings of sound in a practice later called musique concr\`ete.  Later theorized in ``The Guide to the Sonic Object'' ...

\subsection{Modern Collage}
Developments in technology have shifted the affordances of collage practice.  For instance, the Xerox Corporations development of the photocopier led to a surge of do-it-yourself small-publishing houses.  As well, it led to the development of xerox-based collage artists, most notably featured in a bi-monthly xerox-printed zine called \textit{PhotoStatic} which had a peak circulation of 750 copies in the 1980's finally ending with issue 41 in January 1993.

Since then, media has become predominantly digital.  As a result the capabilities of page layout, illustration, and photo editing software have afforded practitioners with faster and more sophisticated methods of composition.  For instance, Adobe Photoshop supports the automatic parsing of an image into object regions that can be individually manipulated enabling artists to finely segment and compose visual media.  For video, Adobe After Effects can similarly segment objects and track features across frames, making dynamic visual collages much easier to do.  

For sound-based collage, ``sampling'', or the process of appropriating existing sound recordings, was born with the birth of the digital sampler.  Sampling, while similar to collage or montage, has one important distinction.  The representation of the fragmented sound is stored as a digital sequence, a series of 1's and 0's that can be perfectly and infinitely copied.  This produces interesting issues in the realm of copyright law, as the pure access of this digital representation is determined to be a copy protected by the content-owners copyright\footnote{These issues are discussed in greater detail later in Section \ref{sec:copyright}, when discussing one output of this thesis, YouTube Smash Up, where the content of YouTube was automatically collaged with the result uploaded each week.}.  Early digital samplers such as the Fairlight CMI and more recent non-linear editors such as Logic and Ableton Live have made multi-track and cut-and-paste operations trivial to accomplish, while visualizing sound waves has made finding relevant parts of an audio file relatively easier than listening to an entire tape reel.  

Lee ``Scratch'' Perry.  King Tubby.  Public Enemy. 

As well, the amount of media has grown, and its access has only become easier with the advent of online archives such as YouTube, Flickr, Soundcloud, Freesound, just to name a few. 

Shifting dialogue from war, communism, capitalism, to property, ownership, copyright.

\subsection{Information Visualization/Auralization}
As collage also works with visualizing large amounts of existing data, the work also shares motivations with information visualization and information auralization, where the data are presented in order to tell a story.  

\subsection{Compression/Cryptography}
In another vein, the work also shares motivations with encoding and decoding, or compression.  The purpose in compression is to take an existing dataset and reduce it down to ``perceptually'' similar information, while removing extraneous information.  

In a related note, this process can also be used for encryption/decryption, where the same process is used to embed meaningful information that can only be deciphered through arcane processes, such as a table look-up or transformation of the data.  

\section{Goals}

This thesis investigates a computational model for automating collage generation.  These parameters effect the two major modular components of the algorithm: attention modeling and representation.   Interaction is focused on the selection of material used in the collage, how the source content is parsed, stored, and for dynamic media, how the collage should be composited over time.  This system affords entirely new experiences around collage, such as the ability to aggregate source content in real-time or experience the collage through an augmented reality headset that mosaics the worlds as it is experienced.  

Encode only parts of a scene that are likely to attract attention... motivate attentional model for dynamic content... 

Develop representation of audio and visual corpus that affords simple interaction to produce different styles...

Fragments of a collage require precarious balance between what is identifiable, i.e. how discernible it is as the original source, and what can be composited, i.e. how it can fit within the greater context. 

\section{Overview}
Attention literature

Representation literature
